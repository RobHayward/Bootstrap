\documentclass{article}\usepackage{graphicx, color}
%% maxwidth is the original width if it is less than linewidth
%% otherwise use linewidth (to make sure the graphics do not exceed the margin)
\makeatletter
\def\maxwidth{ %
  \ifdim\Gin@nat@width>\linewidth
    \linewidth
  \else
    \Gin@nat@width
  \fi
}
\makeatother

\definecolor{fgcolor}{rgb}{0.2, 0.2, 0.2}
\newcommand{\hlnumber}[1]{\textcolor[rgb]{0,0,0}{#1}}%
\newcommand{\hlfunctioncall}[1]{\textcolor[rgb]{0.501960784313725,0,0.329411764705882}{\textbf{#1}}}%
\newcommand{\hlstring}[1]{\textcolor[rgb]{0.6,0.6,1}{#1}}%
\newcommand{\hlkeyword}[1]{\textcolor[rgb]{0,0,0}{\textbf{#1}}}%
\newcommand{\hlargument}[1]{\textcolor[rgb]{0.690196078431373,0.250980392156863,0.0196078431372549}{#1}}%
\newcommand{\hlcomment}[1]{\textcolor[rgb]{0.180392156862745,0.6,0.341176470588235}{#1}}%
\newcommand{\hlroxygencomment}[1]{\textcolor[rgb]{0.43921568627451,0.47843137254902,0.701960784313725}{#1}}%
\newcommand{\hlformalargs}[1]{\textcolor[rgb]{0.690196078431373,0.250980392156863,0.0196078431372549}{#1}}%
\newcommand{\hleqformalargs}[1]{\textcolor[rgb]{0.690196078431373,0.250980392156863,0.0196078431372549}{#1}}%
\newcommand{\hlassignement}[1]{\textcolor[rgb]{0,0,0}{\textbf{#1}}}%
\newcommand{\hlpackage}[1]{\textcolor[rgb]{0.588235294117647,0.709803921568627,0.145098039215686}{#1}}%
\newcommand{\hlslot}[1]{\textit{#1}}%
\newcommand{\hlsymbol}[1]{\textcolor[rgb]{0,0,0}{#1}}%
\newcommand{\hlprompt}[1]{\textcolor[rgb]{0.2,0.2,0.2}{#1}}%

\usepackage{framed}
\makeatletter
\newenvironment{kframe}{%
 \def\at@end@of@kframe{}%
 \ifinner\ifhmode%
  \def\at@end@of@kframe{\end{minipage}}%
  \begin{minipage}{\columnwidth}%
 \fi\fi%
 \def\FrameCommand##1{\hskip\@totalleftmargin \hskip-\fboxsep
 \colorbox{shadecolor}{##1}\hskip-\fboxsep
     % There is no \\@totalrightmargin, so:
     \hskip-\linewidth \hskip-\@totalleftmargin \hskip\columnwidth}%
 \MakeFramed {\advance\hsize-\width
   \@totalleftmargin\z@ \linewidth\hsize
   \@setminipage}}%
 {\par\unskip\endMakeFramed%
 \at@end@of@kframe}
\makeatother

\definecolor{shadecolor}{rgb}{.97, .97, .97}
\definecolor{messagecolor}{rgb}{0, 0, 0}
\definecolor{warningcolor}{rgb}{1, 0, 1}
\definecolor{errorcolor}{rgb}{1, 0, 0}
\newenvironment{knitrout}{}{} % an empty environment to be redefined in TeX

\usepackage{alltt}
\usepackage[colorlinks = true, urlcolor = blue]{hyperref}
\title{Boostrap}
\author{Rob Hayward}
\IfFileExists{upquote.sty}{\usepackage{upquote}}{}
\begin{document}
\maketitle
There are a couple of files that provide an overview of boostrap. The first is  \href{http://danielmarcelino.com/got-bootstrap/}{Daniel Marcelino}

Marcelino creates a function to do the boostrap.  
\begin{knitrout}
\definecolor{shadecolor}{rgb}{0.969, 0.969, 0.969}\color{fgcolor}\begin{kframe}
\begin{alltt}
myboot <- \hlfunctioncall{function}(data, stat, nreps, hist = TRUE) \{
    estimates <- \hlfunctioncall{get}(stat)(data)
    len <- \hlfunctioncall{length}(estimates)
    container <- \hlfunctioncall{matrix}(NA, ncol = len, nrow = nreps)
    nobs <- \hlfunctioncall{nrow}(data)
    \hlfunctioncall{for} (i in 1:nreps) \{
        posdraws <- \hlfunctioncall{ceiling}(\hlfunctioncall{runif}(nobs) * nobs)
        resample <- data[posdraws, ]
        container[i, ] <- \hlfunctioncall{get}(stat)(resample)
    \}
    sds <- \hlfunctioncall{apply}(container, 2, sd)
    \hlfunctioncall{if} (hist == T) \{
        mfrow = \hlfunctioncall{c}(1, 1)
        \hlfunctioncall{frame}()
        \hlfunctioncall{if} (len <= 3) 
            \hlfunctioncall{par}(mfrow = \hlfunctioncall{c}(len, 1))
        \hlfunctioncall{if} ((len > 3) & (len <= 6)) 
            \hlfunctioncall{par}(mfrow = \hlfunctioncall{c}(3, 2))
        \hlfunctioncall{for} (j in 1:len) \hlfunctioncall{hist}(container[, j], main = \hlfunctioncall{paste}(\hlstring{"Estimates for "}, 
            \hlfunctioncall{names}(estimates)[j]), xlab = \hlstring{""})
    \}
    \hlfunctioncall{print}(\hlfunctioncall{rbind}(estimates, sds))
    \hlfunctioncall{return}(\hlfunctioncall{list}(estimation = container, sds = sds))
\}
\end{alltt}
\end{kframe}
\end{knitrout}


\end{document}
